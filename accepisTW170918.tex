%% document class
\documentclass[12pt, a4paper,oneside]{article}

%% global settings
\input{settings/GlobalSettings}

%% specific settings for Harz Epiphytes
\input{settings/HarzSpecific}

% Set Path for Graphics 
\graphicspath{{/home/dimerandra/Documents/Studium/16HarzEpis/results/}} % Specifies the directory where pictures are 



% ===========================================================================================
\begin{document} \doublespacing
% ===========================================================================================


\begin{titlepage}
	\newcommand{\HRule}{\rule{\linewidth}{0.5mm}} % Defines a new command for the horizontal lines, change thickness here
	\center % Center everything on the page
	 %----------------------------------------------------------------------------------------
	
	\textsc{\large - Carl von Ossietzky University of Oldenburg - }\\[.5cm] % Name of your university/college
	\HRule \\[0.15cm]
	{ \huge \bfseries Survey of Accidental Vascular Epiphytes in the Harz Mountains}\\[0.2cm] % Title of your document
	%----------------------------------------------------------------------------------------
	\begin{figure}[!ht]
		\centering
		\includegraphics[width = 1\textwidth, angle = 90] {images/DSC6246.jpg}
		\label{fig:titlefig}
	\end{figure}%
	\vspace{-.8cm}
	%----------------------------------------------------------------------------------------
	\HRule \\[1.2cm]
	%----------------------------------------------------------------------------------------

	{\large
	\begin{tabbing}
		\hspace*{2cm}\=\hspace*{1.5cm}\= \kill
		\raggedleft\emph{Author:} \>  \> Tizian Weichgrebe\\
		\raggedleft\emph{Supervisor:} \>  \> Prof. Dr. Gerhard Zotz\\
		\raggedleft\emph{Date:} \>  \> \today
	\end{tabbing} }
				
	%----------------------------------------------------------------------------------------
	\vfill % Fill the rest of the page with whitespace


\end{titlepage}
% ===========================================================================================
\todo{species names: get the descriptor for all species!}
\tableofcontents
\newpage
% ===========================================================================================
\section*{Abstract}%
This is my great abstract




% ===========================================================================================
\section{Introduction}
As common as epiphytes are in the tropics, it is hard to identify inert characteristics in being an epiphyte. To further define this functional group of plants, it is useful to study the differences of facultative and so-called accidental epiphytes. Which factors enable some terrestrial plants to live accidentally as epiphytes? And which hinder most of terrestrial plants not to? If there are nearly no facultative epiphytes in the temperate zone, epiphytism may be a selective threshold. Especially in the temperate zone, there are distinctively less epiphytes than in the tropics with a nearly irrelevant proportion of the ecosystem biomass. Even though accidental epiphytes in the temperate zones have been evaluated not to be meant to grow on other plants, there is hardly any quantitative background to this hypothesis yet.\\

To quantify (vascular) epiphyte occurrences in the temperate zones, we are going to conduct an ongoing survey in the Harz Mountains. Furthermore, this survey is about to determine in how far temperate epiphytes are just accidental or if there are recurring occurrences of epiphytism which can be beneficial for various reasons. We chose the Harz Mountains for their humid climate with partly old forests. Preferably, we want to reproduce the tree community of the Harz Mountains so to study a possible host specificity of epiphytes which could then give information on niche demands of temperate epiphytes.

Both, host tree identity as well as elevation were anticipated to be important factors for%

For each tree a set of parameters was recorded. Moss cover was roughly estimated due to its function as water storage or for soil accumulation. 
Especially in the lower trunk regions of trees, there was often a transition zone between mosses and soil covered bark. 

% ===========================================================================================
\section{Materials \& Methods}
To display the tree community as well as various elevations of the Harz Mountains, the survey nearly stretched above the whole area of the mountains. By that, we focused mostly on the common tree species but included even some rare as well as \imp{artificial} species to record potential epiphyte load. Since elevation was anticipated to be an important factor  influencing tree structure as well as the potentially epiphytic vegetation, a wide elevation range was examined. However, we focused on higher elevations because the increased humidity was expected to increase the epiphyte load.

% --------------------------------------------------------------------------------------------
\subsection{Tree Stands}
In the following, tree stands and plots are used synonymously. For practical reasons and for a certain unlikelihood of epiphytes to occur, trees with a diameter at breast height (dbh) of less than 5 cm were disregarded. Selection of plots followed linear structures such as streets or was tried to match in parking sites. Following that, it was distinguished between streets, forest roads (still drivable with car), dirt tracks (in forest, enlarged for hiking), beaten paths (in forest, small track). For each of the plots, about 50 trees were selected for examination. If trees were on both sides of a plot, about the same number of trees was taken from each. If there were for any reason less than 50 trees, then even noticeably smaller numbers were taken as well. This was done to include small tree stands with rare species as well. Since humidity was expected to be an if not the most important factor to epiphyte viability, the distance of the tree stands to the next river or lake was estimated. For each plot, the type of structure (different path types), number of trees, elevation (m), coordinates were noted.

% --------------------------------------------------------------------------------------------
\subsection{Measured Parameters}
Following variables were examined for each tree: plot, tree species, dbh (cm), height (m), an estimate of the moss cover (\%), ratio of occupied to empty forks and number as well as species of epiphytes. Noted parameters for examined epiphytes were: tree id, epiphyte species, height in the tree (m), locations on the tree \todo{define classes for locations and substrate} and presence of blossoms or fruits (0/1). If epiphytes were on the trunk itself, they were noted as such. To distinguish between plants that grow up a trunk from below and use this transition zone as well as plants that are truly epiphytic in higher parts of trees, the height of epiphytes in the trees was estimated. Additionally, photos of each tree with epiphytes and of each epiphyte itself were taken for later identification and to find them again in coming surveys.

% --------------------------------------------------------------------------------------------
\subsection{Statistical Analysis}
\todo{sum up the used statistical methods and the r-quatsch...}

The map with plots were created using the \textit{ggmap}-package in R by \textcite{Kahle2013} and the map data was extracted from \textcite{GoogleMaps2017}.

In the following, the $\pm$ symbol is used to display the mean $\pm$ the standard deviation.

% ===========================================================================================
\section{Results}
In total, 34 tree stands were selected along an elevation range of \mRange{102}{807} and with an average of \mErrRange{406}{139}. The spread of plots was biased to the western side of the Harz Mountains (due to higher elevations) but nearly covered their whole area. \todo{topographical map of the Harz? including the plots! and weather?}
%\vskip \quad
\bigskip
\begin{figure}[!h]
%	\ltablewidth
	\includegraphics[width=1\textwidth, angle =0] {images/harzmap_terrain.pdf}
	\label{fig:harzmap}
	\vspace*{-5mm} \caption{Map of the Harz Mountains with studied plots and two additional sites (orange). Labeled plots are sorted for the month of their survey (dark blue = July, light blue = June, violet = September, all in 2016). The topographical map was extracted from the Google Map data; green coloration stands for protected areas. Note that some peripheral parts of the Harz are missing especially of the eastern part.}
\end{figure}


\subsection{Tree Stands}
Even though plots with about 50 trees were preferred, the average plot contained only about 37 trees and the least habited still 17 trees (\autoref{tab:averageplot}). Most of the plots contained between three to nine tree species, with one plot counting 15 species. Epiphytes were found along the whole range of elevation as well as in all plots except for one. On average, there were slightly more epiphytes per plot than trees. However, they showed a much higher variation in the number of individuals (0--122) as well as of species (0--19) within the tree stands. 

\begin{wraptable}[9]{r}{0.5\textwidth}%
	\centering
%	\vspace{-20pt}
	\caption{Mean and standard deviation of individuals and species of trees and epiphytes per plot.}
	\label{tab:averageplot}
		{\small 
		\begin{tabular}{llc}
			\toprule
			\multicolumn{3}{l} {\textbf{Average per plot}} \\
			\midrule
			%Per plot & mean $\pm$ sd &  \\ \hline
			Trees     & individuals   & 37.71 $\pm$ 11.86  \\
			          & species       &  6.32 $\pm$ 2.68   \\
			Epiphytes & individuals   & 42.62 $\pm$ 37.16  \\
			          & species       &  8.71 $\pm$ 5.49   \\ 
			\bottomrule
		\end{tabular} }
%		\vspace{-10pt}
	\end{wraptable}

% as \textcite{Zotz2009} will know, here the results will appear.
In total, the survey included 1282 tree individuals from 35 species and 13 families which differed a lot in their abundance (cf. \autoref{tab:alltrees} for a complete list).  Often, tree species showed clustering around a few single plots and solely eight of them occurred in more than 10 plots (\autoref{tab:commontrees}). Most common was \textit{Acer pseudoplatanus}, which was found in over \prc{80} of the plots representing \prc{28} of all trees. Frequent were as well \textit{Acer platanoides} (in \prc{62} of plots and \prc{12} of individuals) and \textit{Fraxinus excelsior} (\prc{56}, \prc{10}). However, it should be noted that the selection of plots was in no sense representative of the forest structure of the Harz mountains but rather aimed for sites with higher epiphyte abundances.

Of the 35 trees species, 25 had epiphytes on them (\prc{71}). However, the ten remaining tree species made up less than \prc{2} of all trees. In total, epiphytes were found on about \prc{23} of the tree individuals. Most occupied of the tree species was \textit{Salix alba} with \prc{76} and an average number of 6 epiphytes per tree (21 individuals), followed by six other tree species which where occupied in around \prc{40} of the time (5--94 individuals, on average 4--10 epiphytes per tree). The most common tree species had a lower occupation proportion but still around 4--6 epiphytes per individual, like \textit{A. pseudoplatanus} (\prc{24}), \textit{A. platanoides} (\prc{34}) and \textit{F. excelsior} (\prc{28}).
\todo{careful! the mean of epiphytes per trees is not correct, because this includes only occupied trees, not all!}

%\input{tables/commontrees}

\paragraph[Acer ssp.]{\textit{Acer ssp.}}
..further details here... Besides the two common \textit{Acer} species, seven individuals of \textit{A. campestre} were found in two plots.

\paragraph[Picea abies]{\textit{Picea abies}} Since especially high stretches of the Harz Mountains on the western side were and are still used for silviculture, conifers are much more common. Specifically, some of these forests are basically monocultures of \textit{P. abies}. Due to a partial reversal of , parts of these conifer forests are protected areas such as on the highest mountain of the Harz, namely the 'Brocken'. In a prior inspection neither in the economically used nor in the protected forest of \textit{P. abies} any epiphytes were found. For this reason, only three plots with a high percentage of this tree species were included (in elevations between \mRange{307}{630}). In these, \textit{P. abies} made up between \pRange{30}{60} of the trees (with about eight other tree species present). And even though nearly \prc{60} of it's individuals occurred, a single individual of \textit{Oxalis acetosella} at the stem base was the only epiphyte found for \textit{P. abies} in these three plots. 

\paragraph{Angiospermae vs. Gymnospermae}
The epiphytic load not only differed between individual tree species, but this can be partially traced back to a clear distinction between Angio- and Gymnosperms. While 296 of the 1282 surveyed trees were occupied with epiphytes (\prc{23}), this ratio was even slightly higher for Angiosperms (\prc{25}, 286 of 1148 occupied) but much lower for Gymnosperm trees (\prc{7}, 10 of 134 occupied). 


\todo{work on \autoref{tab:allepis}: take out elevation? check the elevational gradients of the epis! Add a line with the total count for the most common genera!}

	
\subsection{Epiphytes}
All of the 15 most important taxa were very common herb genera abundant in the flora of the Harz Mountains like \textit{Geranium}, \textit{Galeopsis} and \textit{Impatiens} (cf. \autoref{tab:commonepis}; cf. \autoref{tab:allepis} for the complete list of taxa). 

\paragraph{Woody Taxa} Most of the taxa were herbaceous with some exceptions of shrub or even tree species. Most prominent of the latter was \textit{Sorbus aucuparia} with 50 individuals. Its largest individual was \mtr{3.5} in size and its highest in a fork in \mtr{8}. Noteworthy were as well the two abundant \textit{Acer} species of which 43 individuals were found as epiphytes. Furthermore, there were \imp{...} other woody taxa growing epiphytic (namely \textit{Abies sp.}, \textit{Carpinus betulus}, \textit{Crataegus monogyna}, \textit{Fraxinus excelsior}, \textit{Picea abies}, \textit{Sambucus nigra}, \textit{Ulmus glabra}, \textit{Vaccinium myrtillus} and one unidentified species of the \textit{Pinaceae}.


\begin{figure}[!h]
%	\ltablewidth
	\includegraphics[width=1\textwidth, angle =0] {graphs/common_hmd.pdf}
	\label{fig_common_overview}
	\caption{Height of epiphyte individuals (m), the moss cover (\%) and the DBH (cm) for the 15 most common epiphyte genera. The genera are sorted in ascending order with the number of found individuals (n). The numbers give the total number of individuals (n) and the number of occurrences on trees (occ) per genus.}
\end{figure}
\todo{redo \autoref{fig_common_overview}, you don't need dbh, doesn't say anything. what about Height:at? some genera are only found quite low and Sorbus only quite high!}

\input{tables/commonepis}


\bigskip
%\begin{figure}[!h]
%	%	\ltablewidth
%	
%	\includegraphics[width=1\textwidth, angle =0] {graphs/ordination/Sub_DCA_orditorp_Epis.pdf}
%	\includegraphics[width=1\textwidth, angle =0] {graphs/ordination/Sub_DCA_orditorp_Trees.pdf}
%	\label{fig:subdca}
%	\vspace*{-5mm} \caption{Detrended correspondence analysis (DCA) of (A) all epiphytes above 0.5 m and (B) their host trees.}
%\end{figure}

\begin{figure}[ht!]
	\centering
	%	\begin{tabular}{@{}p{0.5\linewidth}@{\enspace}p{0.5\linewidth}@{}}
	\begin{tabular}{@{}p{0.5\linewidth}@{\enspace}p{0.5\linewidth}@{}}
		\subfigimg[width=\linewidth,pos=ul,font=\color{black}]{A}{graphs/ordination/Sub_DCA_orditorp_Epis.pdf} &
		\subfigimg[width=\linewidth,pos=ul,font=\color{black}]{B}{graphs/ordination/Sub_DCA_orditorp_Trees.pdf}\\
	\end{tabular}
	\caption{Detrended correspondence analysis (DCA) of (A) all epiphytes above 0.5 m and (B) their host trees.} 
	\label{fig:subdca}
\end{figure}

% ===========================================================================================
\section{Discussion}


\textit{Picea abies} -->  However, due to a new forestation approach, the forest structure will change. Deciduous trees like \textit{Fagus sylvatica} or the  \textit{Acer} species will benefit from this, which will have an influence on the epiphyte community.

% --------------------------------------------------------------------------------------------
\subsection{Further research}
With the conducted epiphyte survey we ....
	\begin{itemize}
	\item Braun-Blanquiet
	\item nearest neighbor
	\item gradient climatic and topographic
	\item focal trees influencing others
	\item just in sense of repeating the whole...
	\end{itemize}

% --------------------------------------------------------------------------------------------
\newpage
\printbibliography[title={References}]
% ===========================================================================================
\section{Appendix}
\setcounter{table}{0}
\renewcommand{\thetable}{A-\arabic{table}}
Here I will include a table of all found taxa (\autoref{tab:allepis}).\\	

% --------------------------------------------------------------------------------------------
\input{tables/allepis}

% --------------------------------------------------------------------------------------------
\input{tables/alltrees}
%\rowgroup{
%\sperow{

\listoftodos

% ===========================================================================================
\end{document}
% ===========================================================================================



