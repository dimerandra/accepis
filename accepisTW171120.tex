%% document class
\documentclass[12pt, a4paper, oneside, draft]{article}

%% global settings
\input{settings/GlobalSettings}

%% specific settings for Harz Epiphytes
\input{settings/HarzSpecific}

% Set Path for Graphics 
\graphicspath{{../figures/}} % Specifies the directory where pictures are 



% ===========================================================================================
\begin{document} \doublespacing
% ===========================================================================================

\input{content/titlepage}

% ===========================================================================================
\begin{abstract}
	This will be my great abstract
\end{abstract}
\newpage

% ===========================================================================================
\todo{species names: get the descriptor for all species!}
\tableofcontents
\newpage
% ===========================================================================================

\section*{Aims}

\begin{outline}
\0 \textbf{Questions which are *NOT* answered:}
	\1 How accidental are the epiphytes in the north-temperate zone?
	\1 Definition of epiphytism and what makes an obligate epiphyte?
		\2 Cannot be answered, since the data won't allow for this.
\0 \textbf{Questions and aims of the following study:}
	\1 How important are accidental epiphytes in the north-temperate zone?
		\2 So far, only single anecdotal observations exist.
		\2 Therefore, aim is to firstly supply reliable quantitative data on occurrence and distribution of these epiphytes.
	\1 Underlying patterns
		\2 Some species more abundant
			\3 Be careful! May well be influenced by surrounding vegetation.
			\3 Therefore search in literature for common species.
			\3 Compare sites for common species?!
		\2 Host specificity, use indirect ways:
			\3 Epiphytes dependent on bifurkations and humus
			\3[\textrightarrow] Some trees are poor host because they are missing these
			\3 What about moss cover as a further factor?
			\3 \emph{Problem with this:} Epis community influenced by site of trees?
	\1 Definition of accidental:
		\2 Some species are accidentally better suited for epiphytic lifestyle
		\2 New speciation of epiphytic taxa not really possible
			\3 Conitnuous geneflow between individuals on tree and below (pollination, seed distribution)
			\3 Seed bank not large enough and epiphytic individuals too few
	\1 Reasons and factors which facilitate epiphytes in the temperates.
		\2 Only qualitative analysis possible and via literature search...		
\end{outline}


So far no quantitative data on vascular epiphytes in north-temperate zone. Only single anecdotal observations available. Therefore, aim is to firstly supply reliable quantitative data on occurrence and distribution of these epiphytes. And secondly, to further describe the specialities of this niche, underlying patterns and similarities between found epiphytic taxa as well as host trees are analized.

\section{Introduction}

\crr{Start with \imp{definition of obligate, facultative \& accidental epiphytes}. Look into Schimper, Benzing \& Ibisch. For the latter two you should ask Gerhard. (Just living on another plant (without paratising it) does include a lot of possibilities for variation.)}

As common as epiphytes are in the tropics, it is hard to identify inert characteristics in being an epiphyte. A definition of epiphytism is usually very broad. For an instance, \textcite{Zotz2016} described epiphytes as \textit{“plants that germinate and root non-parasitically on other plants at all stages of life”}. \imp{Here you should go more into detail about the classification into obligate, facultative \& accidental (citations including: Burns2010, Zotz2016).}

\crr{To further define this functional group of plants, it is useful to study the differences of facultative and so-called accidental epiphytes. Which factors enable some terrestrial plants to live accidentally as epiphytes? And which hinder most of terrestrial plants not to? If there are nearly no facultative epiphytes in the temperate zone \cim (\cite{Zotz2005} ?), epiphytism may be a selective threshold\imp{what do you even mean with this?}. Especially in the temperate zone, there are distinctively less epiphytes than in the tropics with a nearly irrelevant proportion of the ecosystem biomass \cim. Even though accidental epiphytes in the temperate zones have been evaluated not to be meant to grow on other plants, there is hardly any quantitative background to this hypothesis yet.}\\

\crr{To quantify (vascular) epiphyte occurrences in the temperate zones, we are going to conduct an ongoing survey in the Harz Mountains. Furthermore, this survey is about to determine in how far temperate epiphytes are just accidental or if there are recurring occurrences of epiphytism which can be beneficial for various reasons. We chose the Harz Mountains for their humid climate with partly old forests. Preferably, we want to reproduce the tree community of the Harz Mountains so to study a possible host specificity of epiphytes which could then give information on niche demands of temperate epiphytes.}

\imp{Maybe also cite Ibisch1996 with its classification of epiphytism... Just to have  it. You probably don't have to read it.Or better, write Gerhard a mail.)}

Already \textcite[p. 34]{Staeger1908} accounted for the absence of epiphytes on conifers and \textit{Fagus sylvatica}. They proposed that these tree species occur mainly in \imp{intraspecific} stands and filter too much light for epiphytes to grow. Additionally, \textit{F. sylvatica} has a very smooth bark hindering leaf litter and humus accumulation. Moreover, \textcite{Staeger1908} pointed out that epiphytes would favor large, single standing trees with accumulated leaf litter.

\cmb{One explanation for the occurrences of accidental epiphytes is usually the \cmr{Masseneffekt}. If there is enough seed input then there will be at least some seeds germinating. Any arguments pro/contra that? What is with flowering/fruiting plants? Maybe these show that the environment cannot be that hostile?!}

\textcite{Zotz2003} drew the attention to the fact, that understanding accidental epiphytism is vital to understand epiphytism itself. In detail, this contains the questions for the necessary conditions for epiphytic individuals. Moreover, when the environmental conditions are met for single accidentally epiphytic individuals, then which conditions enable some species to live mostly or exclusively as epiphytes? Apparently, the various climatic conditions of the north-temperate zone did not facilitate many obligate epiphytic taxa.

\textcite{Burns2010} argued that terrestrial individuals of obligate epiphytes may constitute a sink population. Accordingly, epiphytic individuals of very common terrestrial temperate species could be the result of a continuous input of their propagules.


\paragraph{Aims}
As there are hardly any obligate or facultative epiphytes in the north-temperate zone, the study of epiphytism is still quite limited for these areas. However, from the above mentioned literature it was frequently recorded that many accidentally epiphytic taxa do exist here. But what is still missing is a comprehensive and thorough \imp{description?} of these accidental epiphytes. This survey aims to lay the foundation to such an undertaking.

Aims of this study are to
\begin{itemize}
	\item[…] get quantitative data on the number of epiphytes in the north-temperate zone.
	\item[…] discuss, if these epiphytic occurring species are accidental, facultative or obligate epiphytes
	\subitem - however, since no further data on terrestrial plant individuals is examined, this cannot be based on the above mentioned definition
	\item[…] discuss, what could cause or facilitate the occurrence of epiphytic individuals in these ecosystems 
\end{itemize}

As \textcite{Zotz2003} summarized, there is a number of (mostly historical) publications which already mention the existence of epiphytic individuals in the north-temperate zone. Since all of these records are descriptive works, the aim of this study is to \imp{supply?!} quantitative data on a larger scale of epiphytism in these ecosystems. Underlying questions are, how (un-)important epiphytes are in these systems and in how far their occurrences are to be understood as accidental. Based on the reports by \imp{cite the historical source here}, I hypothesized that on the one hand epiphytes play a minor role in the north-temperate zone but that on the other hand the term \textit{accidental} is \imp{unangebracht} and based on wrong assumptions. Furthermore, a study of these temperate epiphytes can aid in understanding of epiphytism itself.



\paragraph{Details for later:}
Both, host tree identity as well as elevation were anticipated to be important factors for%
For each tree a set of parameters was recorded. Moss cover was roughly estimated due to its function as water storage or for soil accumulation. 
Especially in the lower trunk regions of trees, there was often a transition zone between mosses and soil covered bark. 


Limitations of this work: Terrestrial plants around trees not examined. Therefore no conclusions  about accidental/facultative epiphytes are possible. However, one can argue that the occuring epiphytic individuals all belong to very common terrestrial species.

% ===========================================================================================
\section{Materials \& Methods} \label{sec:MM}
To display the tree community as well as various elevations of the Harz Mountains, the survey nearly stretched above the whole area of the mountains. By that, we focused mostly on the common tree species but included even some rare as well as \imp{artificial} species to record potential epiphyte load. Since elevation was anticipated to be an important factor  influencing tree structure as well as the potentially epiphytic vegetation, a wide elevation range was examined. However, we focused on higher elevations because the increased humidity was expected to increase the epiphyte load.

	% --------------------------------------------------------------------------------------------
	\subsection{Tree Stands}
	In the following, tree stands and plots are used synonymously. For practical reasons and for a certain unlikelihood of epiphytes to occur, trees with a diameter at breast height (dbh) of less than 5 cm were disregarded. Selection of plots followed linear structures such as streets or was tried to match in parking sites. Following that, it was distinguished between streets, forest roads (still drivable with car), dirt tracks (in forest, enlarged for hiking), beaten paths (in forest, small track). For each of the plots, about 50 trees were selected for examination. If trees were on both sides of a plot, about the same number of trees was taken from each. If there were for any reason less than 50 trees, then even noticeably smaller numbers were taken as well. This was done to include small tree stands with rare species as well. Since humidity was expected to be an if not the most important factor to epiphyte viability, the distance of the tree stands to the next river or lake was estimated. For each plot, the type of structure (different path types), number of trees, elevation (m), coordinates were noted.
	
	% --------------------------------------------------------------------------------------------
	\subsection{Measured Parameters}
	Following variables were examined for each tree: plot, tree species, dbh (cm), height (m), an estimate of the moss cover (\%), ratio of occupied to empty forks and number as well as species of epiphytes. Noted parameters for examined epiphytes were: tree id, epiphyte species, height in the tree (m), locations on the tree \todo{define classes for locations and substrate} and absence (0) or presence of blossoms (1) or fruits (2). If epiphytes were on the trunk itself, they were noted as such. To distinguish between plants that grow up a trunk from below and use this transition zone as well as plants that are truly epiphytic in higher parts of trees, the height of epiphytes in the trees was estimated. Additionally, photos of each tree with epiphytes and of each epiphyte itself were taken for later identification and to find them again in coming surveys.
	
	% --------------------------------------------------------------------------------------------
	\subsection{Statistical Analysis}
	\todo{sum up the used statistical methods and the r-quatsch...}
	
	The map with plots were created using the \textit{ggmap}-package in R by \textcite{Kahle2013} and the map data was extracted from \textcite{GoogleMaps2017}.
	
	In the following, the $\pm$ symbol is used to display the mean $\pm$ the standard deviation.
		
		\paragraph{Multivariate Statistics} \todo{cite multivariate analysis models!}
		To view the epiphyte distribution in dependence of their host trees including environmental factors per tree and plot, ordination methods were applied. For this, mainly the package \textit{vegan} was used \parencite{Oksanen2017}. To determine, if a linear or unimodal model should be applied, a Detrended Correspondence Analysis (DCA) was computed. If the axis length of the first axis in the DCA was larger than \imp{1, really?}, unimodal ordination methods were selected. Afterwards the DCA was compared to a Correspondence Analysis (CA) and chosen over the latter, if the Eigenvalue of the second axis was considerably lower (due to the arch effect). This indirect ordination method was furthermore run as a Detrended Canonical Correspondence Analysis (DCCA) by computing a post-hoc fit for the environmental factors via the \textit{envfit} R~function (included in the \textit{vegan} package). Additionally to the post-hoc fit, a Canonical Correspondence Analyses (CCA) was computed as a direct ordination method. 
		
		the environmental data to allow for statistical testing by a Monte-Carlo permutation analysis. As an additional way of testing the influence of environmental factors, Canonical Correspondence Analyses (CCA) were run as well. Both, the DCCAs and the CCAs were statistically tested by running 
		

% ===========================================================================================
\section{Results}
In total, 34 tree stands were selected along an elevation range of \mRange{102}{807} and with an average of \mErrRange{406}{139}. The spread of plots was biased to the western side of the Harz Mountains (due to higher elevations) but nearly spread over their whole area. \todo{topographical map of the Harz? including the plots! and weather?}

\input{figures/harzmap}
	
	% --------------------------------------------------------------------------------------------
	\subsection{Tree Stands}
	Even though plots with about 50 trees were preferred, the average plot contained only about 37 trees and the least habited still 17 trees (\autoref{tab:averageplot}). Most of the plots contained between three to nine tree species, with one plot counting 15 species. Epiphytes were found along the whole range of elevation as well as in all plots except for one. On average, there were slightly more epiphytes per plot than trees. However, they showed a much higher variation in the number of individuals (0--122) as well as of species (0--19) within the tree stands. 
	
	\input{tables/summary_wraptable}
	
	% as \textcite{Zotz2009} will know, here the results will appear.
	In total, the survey included 1282 tree individuals from 35 species and 13 families which differed a lot in their abundance (cf. \autoref{tab:alltrees} for a complete list).  Often, tree species showed clustering around a few single plots and solely eight of them occurred in more than 10 plots (\autoref{tab:commontrees}). Most common was \textit{Acer pseudoplatanus}, which was found in over \prc{80} of the plots representing \prc{28} of all trees. Frequent were as well \textit{Acer platanoides} (in \prc{62} of plots and \prc{12} of individuals) and \textit{Fraxinus excelsior} (\prc{56}, \prc{10}). However, it should be noted that the selection of plots was in no sense representative of the forest structure of the Harz mountains but rather aimed for sites with higher epiphyte abundances.
	
	Of the 35 trees species, 25 had epiphytes on them (\prc{71}). However, the ten remaining tree species made up less than \prc{2} of all trees. In total, epiphytes were found on about \prc{23} of the tree individuals. Most occupied of the tree species was \textit{Salix alba} with \prc{76} and an average number of 6 epiphytes per tree (21 individuals), followed by six other tree species which where occupied in around \prc{40} of the time (5--94 individuals, on average 4--10 epiphytes per tree). The most common tree species had a lower occupation proportion but still around 4--6 epiphytes per individual, like \textit{A. pseudoplatanus} (\prc{24}), \textit{A. platanoides} (\prc{34}) and \textit{F. excelsior} (\prc{28}).
	\todo{careful! the mean of epiphytes per trees is not correct, because this includes only occupied trees, not all!}
	
	\input{tables/commontrees}
	
		\paragraph[Acer ssp.]{\textit{Acer ssp.}}
		..further details here... Besides the two common \textit{Acer} species, seven individuals of \textit{A. campestre} were found in two plots.
		
		\paragraph[Picea abies]{\textit{Picea abies}} Since especially high stretches of the Harz Mountains on the western side were and are still used for silviculture, conifers are much more common. Specifically, some of these forests are basically monocultures of \textit{P. abies}. Due to a partial reversal of \imp{something missing here!},  parts of these conifer forests are protected areas such as on the highest mountain of the Harz, namely the 'Brocken'. In a prior inspection neither in the economically used nor in the protected forest of \textit{P. abies} any epiphytes were found. For this reason, only three plots with a high percentage of this tree species were included (in elevations between \mRange{307}{630}). In these, \textit{P. abies} made up between \pRange{30}{60} of the trees (with about eight other tree species present). And even though nearly \prc{60} of it's individuals occurred, a single individual of \textit{Oxalis acetosella} at the stem base was the only epiphyte found for \textit{P. abies} in these three plots. 
		
		\paragraph{Angiospermae vs. Gymnospermae}
		The epiphytic load not only differed between individual tree species, but this can be partially traced back to a clear distinction between Angio- and Gymnosperms. While 296 of the 1282 surveyed trees were occupied with epiphytes (\prc{23}), this ratio was even slightly higher for Angiosperms (\prc{25}, 286 of 1148 occupied) but much lower for Gymnosperm trees (\prc{7}, 10 of 134 occupied). 
		
		\todo{work on \autoref{tab:allepis}: take out elevation? check the elevational gradients of the epis! Add a line with the total count for the most common genera!}
	
	% --------------------------------------------------------------------------------------------	
	\subsection{Epiphytes}  
	\imp{You totally forgot to give the most important information!!} Overall, we found epiphytic individuals of \cmr{xxx} species and \cmr{xxx} genera from \cmr{xxx} families. All of the 15 most important taxa were very common herb genera abundant in the flora of the Harz Mountains like \textit{Geranium}, \textit{Galeopsis} and \textit{Impatiens} (cf. \autoref{tab:commonepis}; cf. \autoref{tab:allepis} for the complete list of taxa). 

		\paragraph{Woody Taxa} Most of the taxa were herbaceous with some exceptions of shrub or even tree species. Most prominent of the latter was \textit{Sorbus aucuparia} with 50 individuals. Its largest individual was \mtr{3.5} in size and its highest in a fork in \mtr{8}. Noteworthy were as well the two abundant \textit{Acer} species of which 43 individuals were found as epiphytes. Furthermore, there were \imp{...} other woody taxa growing epiphytic (namely \textit{Abies sp.}, \textit{Carpinus betulus}, \textit{Crataegus monogyna}, \textit{Fraxinus excelsior}, \textit{Picea abies}, \textit{Sambucus nigra}, \textit{Ulmus glabra}, \textit{Vaccinium myrtillus} and one unidentified species of the \textit{Pinaceae}.
		
		\input{tables/commonepis}
	
	
		\paragraph{Epiphyte Location}
		We could identify different patterns of epiphyte distribution within the trees (\ref{fig:heightclasses}). Sorted into four height classes (\autoref{sec:MM}), some trees hosted nearly no epiphytes above \prc{10} of their height (\autoref{fig:heightclasses} E). Only \textit{A. pseudoplatanus} hosted more epiphytes above than below \prc{10} of its height (\autoref{fig:heightclasses} D).
		
		\input{figures/height_classes}
	
		Viewed from the epiphytic perspective, most species grew mainly quite low in the tree at the stem base (\autoref{fig:heightclasses} A). However, some species were more commonly found above \mtr{.5}. One example of these is \textit{S. aucuparia}, of which \prc{90} (n = 50) of the individuals grew between \pRange{10}{50} of the tree's height. Other genera, which were more commonly found between \pRange{10}{50} of the height were \textit{Galeopsis} (\prc{64}, n = 47), \textit{Rubus} (\prc{67}, n = 12) and the species \textit{A. pseudoplatanus} (\prc{67}, n = 6)
		
	% --------------------------------------------------------------------------------------------
	\subsection{Multivariate Statistics}
	In search of underlying patterns of host specificity and the influence of environmental factors, multivariate statistics namely ordination methods were applied to the data set. However, no clear patterns submerged neither in the detrended nor in the canonical correspondence analyses \imp{Add statistics here!}. When the same methods were used for a subset containing only epiphytes above \mtr{0.5}, 
	
	\input{figures/multivariate_statistics}

% ===========================================================================================
\section{Discussion}
\textit{Picea abies} -->  However, due to a new forestation approach, the forest structure will change. Deciduous trees like \textit{Fagus sylvatica} or the  \textit{Acer} species will benefit from this, which will have an influence on the epiphyte community.

The vertical distribution of the epiphytes behaved accordingly to the findings of \textcite{Zotz2003} at least for the three common species.

\imp{Staeger1908 included a list with all found taxa. Compare with your own.} As \textcite[pp.~52]{Staeger1908} compared the epiphytic assemblages in two Swiss Alpine valleys hundred years ago, they found in total 89 accidentally epiphytic plant species. Similarly we counted at least \cmr{xxx} species. However, both of their examined valleys shared only 19 species with each other. And compared to the species found in the Harz mountains, these mostly overlap. This is probably due to the fact that most of these were overall very common terrestrial species. Still, one has to consider that their examination of the epiphytic flora was much like an anecdote compared to this methodological approach. Therefore, 

One important point is the differentiation between this and that.


	% --------------------------------------------------------------------------------------------
	\subsection{Further research}
	With the conducted epiphyte survey we ....
		\begin{itemize}
		\item Braun-Blanquiet
		\item nearest neighbor
		\item gradient climatic and topographic
		\item focal trees influencing others
		\item just in sense of repeating the whole...
		\end{itemize}

% --------------------------------------------------------------------------------------------
\newpage
\printbibliography[title={References}]
% ===========================================================================================
\section{Appendix}
	\setcounter{table}{0}
	\renewcommand{\thetable}{A-\arabic{table}}
	Here I will include a table of all found taxa (\autoref{tab:allepis}).\\	
	
	% --------------------------------------------------------------------------------------------
	\input{tables/allepis}
	
	% --------------------------------------------------------------------------------------------
	\input{tables/alltrees}
	%\rowgroup{
	%\sperow{

\listoftodos

% ===========================================================================================
\end{document}
% ===========================================================================================



