%% document class
\documentclass[12pt, a4paper, oneside, draft]{scrartcl}

%% global settings
\input{settings/GlobalSettings}

%% specific settings for Harz Epiphytes
\input{settings/HarzSpecific}

%% further citation style settings
\input{settings/CitationStyle}

% Set Path for Graphics 
\graphicspath{{../figures/}} % Specifies the directory where figures/pictures are 

\setkomafont{sectioning}{\rmfamily\scshape}



% ===========================================================================================
\begin{document} \doublespacing
% ===========================================================================================

\input{content/titlepage}

% ===========================================================================================

\input{content/abstract}
\newpage

% ===========================================================================================
\todo{species names: get the descriptor for all species! Maybe run ThePlantList-Function in R over it...}
%\tableofcontents
%\newpage
% ===========================================================================================

\section{Introduction}

	\subsubsection*{Definition of obligate, facultative \& accidental epiphytes}
	Speaking of terrestrial vascular epiphytes, one can argue that the term is not yet sufficiently defined. Many authors give a very broad definition of epiphytism as plants growing non-parasitically on other plants \parencite[e.g.][]{Zotz2016, Burns2010, Benzing2012, Brown1948, Schimper1903}. Moreover, many authors have claimed that epiphytism is not a phenomenon isolated from plants which grow on the ground but rather a spectrum with many mixed forms \parencite[e.g.][]{Zotz2016}. Accordingly, species with only a few epiphytic individuals are not to be considered epiphytic species but accidental. To clarify this gradient of epiphytism, some authors introduced a classification into obligate, facultative or accidental epiphytes \parencite{Ibisch1996, Benzing2004}. 
	
	Based on the quantitative composition of the respective species, \textcite{Ibisch1996} classified species dependent on the proportion of individuals living arboreally. With up to \prc{5} epiphytical individuals, species are considered accidental and with more than \prc{95} obligate, respectively. All species not fitting these two groups, are categorized as facultative which are equally well able to live on the ground and epiphytically  \parencite{Ibisch1996}. However, this classification is only a theoretical consideration and \textcite{Burns2010} argued that this method would be arbitrary for different populations of species spanning various locations. Therefore, they proposed that a species is facultatively epiphytic if its arboreal population is predicted by its population on the forest floor \parencite{Burns2010}. Following that, species would be accidental if they occur more frequently than expected on the ground and obligate if more frequently on trees, respectively.
	
	One explanation for the occurrences of accidental epiphytes is usually a quantitative effect \cim. If there is enough seed input then there will be at least some seeds germinating. Therefore, species common on the ground should provide more epiphytical individuals than less common species. A second factor could be the mode of propagation where the epiphytic location should favor species with propagules most suited for distribution on trees. 
	
	\crr{However, what the sheer existence and fruiting \cim shows is that the north-temperate environment cannot be that hostile.}
	
	\crr{To understand the epiphytic functional group of plants, it is useful to study the differences of facultative and the so-called accidental epiphytes. Which factors enable some terrestrial plants to live accidentally as epiphytes? And which hinder most of terrestrial plants not to?}
	
	
	
	\textcite{Zotz2003} drew the attention to the fact, that understanding accidental epiphytism is vital to understand epiphytism itself. In detail, this contains the questions for the necessary conditions for epiphytic individuals. Moreover, when the environmental conditions are met for single accidentally epiphytic individuals, then which conditions enable some species to live mostly or exclusively as epiphytes? Apparently, the various climatic conditions of the north-temperate zone did not facilitate many obligate epiphytic taxa.
	
	\textcite{Burns2010} argued that terrestrial individuals of obligate epiphytes may constitute a sink population. Accordingly, epiphytic individuals of very common terrestrial temperate species could be the result of a continuous input of their propagules.
	
	\crr{Work of this study}
	
	To quantify (vascular) epiphyte occurrences in the temperate zones, we are going to conduct an ongoing survey in the Harz Mountains. Furthermore, this survey is about to determine in how far temperate epiphytes are just accidental or if there are recurring occurrences of epiphytism which can be beneficial for various reasons \imp{(nah)}. We chose the Harz Mountains for their humid climate with partly old forests. Preferably, we want to reproduce the tree community of the Harz Mountains so to study a possible host specificity of epiphytes which could then give information on niche demands of temperate epiphytes.
	
	
	\subsubsection*{Records of north-temperate epiphytes}
	Especially in the temperate zone, there are distinctively less epiphytes than in the tropics with a nearly irrelevant proportion of the ecosystem biomass \cim. Therefore, epiphytism may be a selective threshold \imp{what do you even mean with this?}. Even though accidental epiphytes in the temperate zones have been evaluated not to be meant to grow on other plants, there is hardly any quantitative background to this hypothesis yet.
	\label{key}	
	As described by \textcite{Schimper1888} as well as \textcite{Gentry1987a}, both temperate hemispheres differ in their abundance and the type of epiphytes. Some species rich spots in the south like New Zealand and the southern Himalayan Mountains are quite exceptional \parencite[see ZotzBibliography]{Schimper1888, Hofstede2001}. Furthermore, \textcite{Burns2010} argued that the composition of epiphytes is different for both temperate hemispheres. While the southern consists more of obligate epiphytes, the north mostly comprises facultative epiphytes near the floor. Regarding the study's location, this work therefore mainly focuses on epiphytes of the north-temperate hemisphere with a limited validity for the south.
	
	\crr{Should I discuss the limits of epiphytes, such as frost or drought?! If so, see Zotz' Bibliography... There he also discusses prevalence of conifers, which hinder epiphytes (but does not give any references for that).}
	
	As \textcite{Zotz2003} summarized, many studies on accidental epiphytes in the north-temperate zone (mainly Europe) were already published a hundred years ago. However, since most of them are in an anecdotal form, these studies can only describe common accidentally epiphytic species but neither their importance nor their sheer numbers. From these early authors \parencite[e.g.][]{Beyle1903, Beyer1896, Staeger1912,Magnin1895,Loew1892} (see \parencite{Zotz2003a} for a full bibliography), especially \textcite{Staeger1908} conducted a larger work on epiphytism comparing the species composition of two valleys in the Swiss Alps.
	
	To compare epiphytically growing individuals with, especially studies of \textcite{Staeger1908} and \textcite{Loew1892} are quite suitable due to their higher level of detail. For example, \textcite[p. 34]{Staeger1908} already accounted for the absence of epiphytes on conifers and \textit{Fagus sylvatica}. They proposed that these tree species occur mainly in \imp{intraspecific} stands and filter too much light for epiphytes to grow. Additionally, \textit{F. sylvatica} has a very smooth bark hindering leaf litter and humus accumulation. Moreover, \textcite{Staeger1908} pointed out that epiphytes would favor large, single standing trees with accumulated leaf litter.
	
	Additionally to the work on temperate epiphytes in general, some authors specifically examined the flora on pollard trees like willows usually hosting a larger number of epiphytes than non pruned trees \parencite{Willis1893, Geisenheyner1895, Carriere1977, Bolle1891, Loew1892, Steenis1928, Steenis1925}.
	
	\textcite{Ochsner1927} described quite detailed in how far host trees and environmental factors influence the prevalence of non-vascular epiphytes. 
	
	
	
	
	\subsubsection*{Aims \& Questions}
		From the above mentioned studies it is clear that it is still uncertain, how important these accidental epiphytic individuals are in the north-temperate zone. Based on that, this study determines the dimension of epiphytism in a certain area of this climatic zone and aims to supply a quantitative data on occurrence and distribution of these epiphytes. Moreover, it compares underlying patterns of common epiphytic species and host specificity with observations by above cited research. The definition of \emph{accidental }is shortly discussed regarding the most commonly found taxa. However, since no individuals on the ground were evaluated, this cannot be integrated into the classification by \textcite{Ibisch1996}, as \textcite{Burns2010} did.
		
		\subsubsection*{Hypotheses}
		Hypotheses of this work are: 1. For some species so many epiphytic individuals are found that one cannot speak of accidental.
		
		
	
	As there are hardly any obligate or facultative epiphytes in the north-temperate zone, the study of epiphytism is still quite limited for these areas. However, from the above mentioned literature it was frequently recorded that many accidentally epiphytic taxa do exist here. But what is still missing is a comprehensive and thorough \imp{description?} of these accidental epiphytes. This survey aims to lay the foundation to such an undertaking.
	
	Aims of this study are to
	\begin{itemize}
		\item[…] get quantitative data on the number of epiphytes in the north-temperate zone.
		\item[…] discuss, if these epiphytic occurring species are accidental, facultative or obligate epiphytes
		\subitem - however, since no further data on terrestrial plant individuals is examined, this cannot be based on the above mentioned definition
		\item[…] discuss, what could cause or facilitate the occurrence of epiphytic individuals in these ecosystems 
	\end{itemize}
	
	As \textcite{Zotz2003} summarized, there is a number of (mostly historical) publications which already mention the existence of epiphytic individuals in the north-temperate zone. Since all of these records are descriptive works, the aim of this study is to \imp{supply?!} quantitative data on a larger scale of epiphytism in these ecosystems. Underlying questions are, how (un-)important epiphytes are in these systems and in how far their occurrences are to be understood as accidental. Based on the reports by \imp{cite the historical source here}, I hypothesized that on the one hand epiphytes play a minor role in the north-temperate zone but that on the other hand the term \textit{accidental} is \imp{unangebracht} and based on wrong assumptions. Furthermore, a study of these temperate epiphytes can aid in understanding of epiphytism itself.
	
	
	
	\paragraph{Details for later:}
	Both, host tree identity as well as elevation were anticipated to be important factors for%
	For each tree a set of parameters was recorded. Moss cover was roughly estimated due to its function as water storage or for soil accumulation. 
	Especially in the lower trunk regions of trees, there was often a transition zone between mosses and soil covered bark. 
	
	
	Limitations of this work: Terrestrial plants around trees not examined. Therefore no conclusions  about accidental/facultative epiphytes are possible. However, one can argue that the occuring epiphytic individuals all belong to very common terrestrial species.

% ===========================================================================================
\section{Materials \& Methods} \label{sec:MM}
To display the tree community as well as various elevations of the Harz Mountains, the survey nearly stretched above the whole area of the mountains. By that, we focused mostly on the common tree species but included even some rare as well as \imp{artificial} species to record potential epiphyte load. Since elevation was anticipated to be an important factor  influencing tree structure as well as the potentially epiphytic vegetation, a wide elevation range was examined. However, we focused on higher elevations because the increased humidity was expected to increase the epiphyte load.

	% --------------------------------------------------------------------------------------------
	\subsection{Tree Stands}
	In the following, tree stands and plots are used synonymously. For practical reasons and for a certain unlikelihood of epiphytes to occur, trees with a diameter at breast height (dbh) of less than 5 cm were disregarded. Selection of plots followed linear structures such as streets or was tried to match in parking sites. Following that, it was distinguished between streets, forest roads (still drivable with car), dirt tracks (in forest, enlarged for hiking), beaten paths (in forest, small track). For each of the plots, about 50 trees were selected for examination. If trees were on both sides of a plot, about the same number of trees was taken from each. If there were for any reason less than 50 trees, then even noticeably smaller numbers were taken as well. This was done to include small tree stands with rare species as well. Since humidity was expected to be an if not the most important factor to epiphyte viability, the distance of the tree stands to the next river or lake was estimated. For each plot, the type of structure (different path types), number of trees, elevation (m), coordinates were noted.
	
	% --------------------------------------------------------------------------------------------
	\subsection{Measured Parameters}
	Following variables were examined for each tree: plot, tree species, dbh (cm), height (m), an estimate of the moss cover (\%), ratio of occupied to empty forks and number as well as species of epiphytes. Noted parameters for examined epiphytes were: tree id, epiphyte species, height in the tree (m), locations on the tree \todo{define classes for locations and substrate} and absence (0) or presence of blossoms (1) or fruits (2). If epiphytes were on the trunk itself, they were noted as such. To distinguish between plants that grow up a trunk from below and use this transition zone as well as plants that are truly epiphytic in higher parts of trees, the height of epiphytes in the trees was estimated. Additionally, photos of each tree with epiphytes and of each epiphyte itself were taken for later identification and to find them again in coming surveys.
	
	% --------------------------------------------------------------------------------------------
	\subsection{Statistical Analysis}
	\todo{sum up the used statistical methods and the r-quatsch...}
	
	The map with plots were created using the \textit{ggmap}-package in R by \textcite{Kahle2013} and the map data was extracted from \textcite{GoogleMaps2017}.
	
	In the following, the $\pm$ symbol is used to display the mean $\pm$ the standard deviation.
		
		\paragraph{Multivariate Statistics} \todo{cite multivariate analysis models!}
		To view the epiphyte distribution in dependence of their host trees including environmental factors per tree and plot, ordination methods were applied. For this, mainly the package \textit{vegan} was used \parencite{Oksanen2017}. To determine, if a linear or unimodal model should be applied, a Detrended Correspondence Analysis (DCA) was computed. If the axis length of the first axis in the DCA was larger than \imp{1, really?}, unimodal ordination methods were selected. Afterwards the DCA was compared to a Correspondence Analysis (CA) and chosen over the latter, if the Eigenvalue of the second axis was considerably lower (due to the arch effect). This indirect ordination method was furthermore run as a Detrended Canonical Correspondence Analysis (DCCA) by computing a post-hoc fit for the environmental factors via the \textit{envfit} R~function (included in the \textit{vegan} package). Additionally to the post-hoc fit, a Canonical Correspondence Analyses (CCA) was computed as a direct ordination method. 
		
		the environmental data to allow for statistical testing by a Monte-Carlo permutation analysis. As an additional way of testing the influence of environmental factors, Canonical Correspondence Analyses (CCA) were run as well. Both, the DCCAs and the CCAs were statistically tested by running 
		

% ===========================================================================================
\section{Results}
In total, 34 tree stands were selected along an elevation range of \mRange{102}{807} and with an average of \mErrRange{406}{139}. The spread of plots was biased to the western side of the Harz Mountains (due to higher elevations) but nearly spread over their whole area. \todo{topographical map of the Harz? including the plots! and weather?}

\input{figures/harzmap}
	
	% --------------------------------------------------------------------------------------------
	\subsection{Tree Stands}
	Even though plots with about 50 trees were preferred, the average plot contained only about 37 trees and the least habited still 17 trees (\autoref{tab:averageplot}). Most of the plots contained between three to nine tree species, with one plot counting 15 species. Epiphytes were found along the whole range of elevation as well as in all plots except for one. On average, there were slightly more epiphytes per plot than trees. However, they showed a much higher variation in the number of individuals (0--122) as well as of species (0--19) within the tree stands. 
	
	\input{tables/summary_wraptable}
	
	% as \textcite{Zotz2009} will know, here the results will appear.
	In total, the survey included 1282 tree individuals from 35 species and 13 families which differed a lot in their abundance (cf. \autoref{tab:alltrees} for a complete list).  Often, tree species showed clustering around a few single plots and solely eight of them occurred in more than 10 plots (\autoref{tab:commontrees}). Most common was \textit{Acer pseudoplatanus}, which was found in over \prc{80} of the plots representing \prc{28} of all trees. Frequent were as well \textit{Acer platanoides} (in \prc{62} of plots and \prc{12} of individuals) and \textit{Fraxinus excelsior} (\prc{56}, \prc{10}). However, it should be noted that the selection of plots was in no sense representative of the forest structure of the Harz mountains but rather aimed for sites with higher epiphyte abundances.
	
	Of the 35 trees species, 25 had epiphytes on them (\prc{71}). However, the ten remaining tree species made up less than \prc{2} of all trees. In total, epiphytes were found on about \prc{23} of the tree individuals. Most occupied of the tree species was \textit{Salix alba} with \prc{76} and an average number of 6 epiphytes per tree (21 individuals), followed by six other tree species which where occupied in around \prc{40} of the time (5--94 individuals, on average 4--10 epiphytes per tree). The most common tree species had a lower occupation proportion but still around 4--6 epiphytes per individual, like \textit{A. pseudoplatanus} (\prc{24}), \textit{A. platanoides} (\prc{34}) and \textit{F. excelsior} (\prc{28}).
	\todo{careful! the mean of epiphytes per trees is not correct, because this includes only occupied trees, not all!}
	
	\input{tables/commontrees}
	
		\paragraph[Acer ssp.]{\textit{Acer ssp.}}
		..further details here... Besides the two common \textit{Acer} species, seven individuals of \textit{A. campestre} were found in two plots.
		
		\paragraph[Picea abies]{\textit{Picea abies}} Since especially high stretches of the Harz Mountains on the western side were and are still used for silviculture, conifers are much more common. Specifically, some of these forests are basically monocultures of \textit{P. abies}. Due to a partial reversal of \imp{something missing here!},  parts of these conifer forests are protected areas such as on the highest mountain of the Harz, namely the \emph{Brocken}. In a prior inspection neither in the economically used nor in the protected forest of \textit{P. abies} any epiphytes were found. For this reason, only three plots with a high percentage of this tree species were included (in elevations between \mRange{307}{630}). In these, \textit{P. abies} made up between \pRange{30}{60} of the trees (with about eight other tree species present). And even though nearly \prc{60} of it's individuals occurred, a single individual of \textit{Oxalis acetosella} at the stem base was the only epiphyte found for \textit{P. abies} in these three plots. 
		
		\paragraph{Angiospermae vs. Gymnospermae}
		The epiphytic load not only differed between individual tree species, but this can be partially traced back to a clear distinction between Angio- and Gymnosperms. While 296 of the 1282 surveyed trees were occupied with epiphytes (\prc{23}), this ratio was even slightly higher for Angiosperms (\prc{25}, 286 of 1148 occupied) but much lower for Gymnosperm trees (\prc{7}, 10 of 134 occupied). 
		
		\todo{work on autoref(tab:allepis): take out elevation? check the elevational gradients of the epis! Add a line with the total count for the most common genera!}
	
	% --------------------------------------------------------------------------------------------	
	\subsection{Epiphytes}  
	\imp{You totally forgot to give the most important information!!} Overall, we found epiphytic individuals of \cmr{xxx} species and \cmr{xxx} genera from \cmr{xxx} families. All of the 15 most important taxa were very common herb genera abundant in the flora of the Harz Mountains like \textit{Geranium}, \textit{Galeopsis} and \textit{Impatiens} (cf. \autoref{tab:commonepis}; cf. \autoref{tab:allepis} for the complete list of taxa). 

		\paragraph{Woody Taxa} Most of the taxa were herbaceous with some exceptions of shrub or even tree species. Most prominent of the latter was \textit{Sorbus aucuparia} with 50 individuals. Its largest individual was \mtr{3.5} in size and its highest in a fork in \mtr{8}. Noteworthy were as well the two abundant \textit{Acer} species of which 43 individuals were found as epiphytes. Furthermore, there were \imp{...} other woody taxa growing epiphytic (namely \textit{Abies sp.}, \textit{Carpinus betulus}, \textit{Crataegus monogyna}, \textit{Fraxinus excelsior}, \textit{Picea abies}, \textit{Sambucus nigra}, \textit{Ulmus glabra}, \textit{Vaccinium myrtillus} and one unidentified species of the \textit{Pinaceae}.
		
		\input{tables/commonepis}
	
	
		\paragraph{Epiphyte Location}
		We could identify different patterns of epiphyte distribution within the trees (\ref{fig:heightclasses}). Sorted into four height classes (\autoref{sec:MM}), some trees hosted nearly no epiphytes above \prc{10} of their height (\autoref{fig:heightclasses} E). Only \textit{A. pseudoplatanus} hosted more epiphytes above than below \prc{10} of its height (\autoref{fig:heightclasses} D).
		
		\input{figures/height_classes}
	
		Viewed from the epiphytic perspective, most species grew mainly quite low in the tree at the stem base (\autoref{fig:heightclasses} A). However, some species were more commonly found above \mtr{.5}. One example of these is \textit{S. aucuparia}, of which \prc{90} (n = 50) of the individuals grew between \pRange{10}{50} of the tree's height. Other genera, which were more commonly found between \pRange{10}{50} of the height were \textit{Galeopsis} (\prc{64}, n = 47), \textit{Rubus} (\prc{67}, n = 12) and the species \textit{A. pseudoplatanus} (\prc{67}, n = 6)
		
	% --------------------------------------------------------------------------------------------
	\subsection{Multivariate Statistics}
	In search of underlying patterns of host specificity and the influence of environmental factors, multivariate statistics namely ordination methods were applied to the data set. However, no clear patterns submerged neither in the detrended nor in the canonical correspondence analyses \imp{Add statistics here!}. When the same methods were used for a subset containing only epiphytes above \mtr{0.5}, 
	
	\input{figures/multivariate_statistics}

% ===========================================================================================
\section{Discussion}
\textit{Picea abies} -->  However, due to a new forestation approach, the forest structure will change. Deciduous trees like \textit{Fagus sylvatica} or the  \textit{Acer} species will benefit from this, which will have an influence on the epiphyte community.

The vertical distribution of the epiphytes behaved accordingly to the findings of \textcite{Zotz2003} at least for the three common species.

\imp{Staeger1908 included a list with all found taxa. Compare with your own.} As \textcite[pp.~52]{Staeger1908} compared the epiphytic assemblages in two Swiss Alpine valleys hundred years ago, they found in total 89 accidentally epiphytic plant species. Similarly we counted at least \cmr{xxx} species. However, both of their examined valleys shared only 19 species with each other. And compared to the species found in the Harz mountains, these mostly overlap. This is probably due to the fact that most of these were overall very common terrestrial species. Still, one has to consider that their examination of the epiphytic flora was much like an anecdote compared to this methodological approach. Therefore, 

Already \textcite{Staeger1912} showed that \textit{Acer pseudoplatanus} host a rich epiphyte flora and they explain this with its manner to store large quantities of humus, organic matter and moss covers.  \textcite{Staeger1912} even mentioned that they found earthworms in isolated moss cushions at the stem.

	% --------------------------------------------------------------------------------------------
	\subsection{Further research}
	With the conducted epiphyte survey we ....
		\begin{itemize}
		\item Braun-Blanquiet
		\item nearest neighbor
		\item gradient climatic and topographic
		\item focal trees influencing others
		\item just in sense of repeating the whole...
		\end{itemize}

% --------------------------------------------------------------------------------------------
\newpage
%\printbibliography[title={References}]
\begin{multicols}{2}[\printbibheading]
	\sloppy
%	\raggedright
	\printbibliography[heading=none]
\end{multicols}
% ===========================================================================================
\newpage
\section*{Appendix}
	\setcounter{table}{0}
	\renewcommand{\thetable}{A-\arabic{table}}
	Here I will include a table of all found taxa (\autoref{tab:allepis}).\\	
	
	% --------------------------------------------------------------------------------------------
%	\input{tables/allepis}
	
	% --------------------------------------------------------------------------------------------
%	\input{tables/alltrees}

\listoftodos

% ===========================================================================================
\end{document}
% ===========================================================================================



